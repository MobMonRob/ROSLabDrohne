\newSec[HWParrot]{Parrot Drohne}{2}

Um die Problematik der COEX Drohne zu umgehen, steigt diese \Arbeit\ auf die Drohne um, welche die Idee für diese Studienarbeit ergeben hat.
Hierbei handelt es sich um die Drohne \Ar. Nachfolgend soll die Drohne und die eingebaute Sensorik näher bechrieben werden.






Parrot AR.Drohne 2.0

Laut hersteller folgende Sensorik:

Ultraschall-Sensoren für Bodenabstand

Kamera unten

Kamera vorn


Magnetometer

Beschleunigungssensoren







Sobald IB mit dieser Drohne möglich, wird hier mehr beschrieben.

Falls auch das nicht geht, dann halt Gezabo :D





\newSec{Geometrie}{3}


\newSec[Geom-Rot]{Anordnung der Rotoren}{4}


X




\newSec[Control]{Control Stack}{3}

\missing[Bezeichnung auf deutsch?]



\missing[Drohne ist nur Client]

\missing[Nutzung von ardrone\_autonomy-Package]




\newSec{Sensorik}{3}



\newSec{Gyroskop}{4}



\newSec{Magnetometer}{4}



\newSec[Ultraschall]{Ultraschall-Abstandsmessung}{4}



\newSec{Bodenkamera}{4}



\newSec{Frontkamera}{4}










