\newSec[HWParrot]{Parrot Drohne}{2}

Um die Problematik der COEX Drohne zu umgehen, steigt diese \Arbeit\ auf die Drohne um, welche die Idee für diese Studienarbeit ergeben hat.
Hierbei handelt es sich um die Drohne \Ar. Nachfolgend soll die Drohne und die eingebaute Sensorik näher bechrieben werden.



\newSec{Geometrie}{3}


\newSec[parrotGeom]{Anordnung der Rotoren}{4}
\missing

X


\newSec[parrotSize]{Abmessungen}{4}
\missing







\newSec[Control]{Control Stack}{3}

\missing[Bezeichnung auf deutsch?]



\missing[Drohne ist nur Client]

\missing[Nutzung von ardrone\_autonomy-\Pack]




\newSec{Sensorik}{3}


\begin{itemize}
\item Beschleunigungssensorik
\item Magnetometer
\item Ultraschall-Abstandsmessung zum Boden
\item Frontkamera
\item Bodenkamera
\end{itemize}







\newSec[parrotROS]{Interaktion mittels \ROS}{3}



\newSec[parrotAutonomy]{Treiber \pAuto}{4}
Für die Ansteuerung der Drohne \Ar\ existiert eine Treiber, welche die initialisierung und die Kommunukation mit der Drohne anbietet. Als \ROS-seitige Schnittstelle werden verschiedene \Topic[s] und \Service[s] angeboten, welche nachfolgend näher beschrieben werden sollen.


\newSec[parrotTopics]{Topics}{4}




\missing[Bild von rqt-graph]









\newSec{\rTopic{cmd_vel}}{5}
Nur geschwindigkeits-Daten. Keine tatsächliche Übergabe von Neigungswinkeln und so ODER??
\missing\










\newSec[parrotService]{Services}{4}





\newSec[parrotParams]{Parameter}{4}









