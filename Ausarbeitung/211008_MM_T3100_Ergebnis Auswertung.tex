\FloatBarrier
\newSec[Result]{Auswertung}{2}


Aus der Berechnung der Position in z-Richtung im Vergleich mit den gemessenen Daten (siehe \refImg{fig:FlightPos}) lässt sich zeigen, dass der Ansatz grundsätzlich korrekt gewählt wurde.\\
Da die dieser Auswertung zugrundeliegenden Daten mittel der \textit{Hover}-Funkationalität der \Ar\ (siehe \refCap{parrotTopicsCMD}) aufgezeichnet wurden, ist lediglich eine geringe Abweichung in x- und y-Richtung zu erwarten. Diese kann lediglich durch die empirische Erkenntnisse, nicht durch gelegbare Daten, untermauert werden.
Im Zuge dieser \Arbeit\ konnte keine Begründung für die in \refImg{fig:FlightPos} ersichtlichen Abweichungen der Position in x- und y-Richtung vom lokalen Nullpunkt gefunden werden. Da auch eine sehr geringe Kalibrierung der \CodeClass{PoseBuildable} durchgeführte wird (vgl. \refImg{fig:FlightPoseVel}), ist davon auszugehen, dass die Abweichungen eine tiefergehende Begründung verlangen. Auch der Aufruf der \CodeMeth{reset()} für die beiden Ausrichtungen nach Änderungen der StatusID zu dem Wert 3 (\textit{Flying}) oder dem Wert 4 (\textit{Hovering}) konnte den Verlauf der berechneten Position in der horizontalen Ebene nicht zu einem zufriedenstellenden Ergebnis führen.

\begin{figure}[ht!]
\vspace{0.25cm}
\begin{center}
\fbox{\includegraphics[width=12cm]{Pictures/TestFlight Position Med1 Avg1 Off50 Calib1.png}}
\caption{Testflug Signalverarbeitung: Aufarbeitung $a_z$}
\label{fig:FlightPos}
\end{center}

\vspace{0.25cm}
\refImgShort{fig:FlightPos} zeigt die von dem eingesetzten Programm errechneten Position des \Quad[s] \Ar\ und als Vergleichswert die vom Ultraschall-Sensor gemessenen Abstand zum Untergrund.
\end{figure}





\missing[Etwas zum Thema ungeeignete Initialsensoren. Ungenau oder so...]












