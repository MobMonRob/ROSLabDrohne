\newSec{Erweiterungen}{2}

Dieses Kapitel soll beschreiben, welche weiteren Sensoren eingesetzt werden können, um die Genauigkeit berechneten der Pose zu erhöhen.\\
Da es sich bei den möglichen Erweiterungen um eine Fülle von Varianten handelt, sollen diese jeweils nur knapp beschrieben oder lediglich namentlich genannt werden.



\newSec{Weitere Ansätze der Signalverarbeitung / Pose-Berechnung}{3}





\newSec{Interne Sensoren}{3}

\newSec{Höherwertige Initialsensoren}{4}


\newSec{Magnetometer}{4}



\newSec{Externe Sensoren}{3}


\newSec{Bodenkamera}{4}


\newSec{Frontkamera}{4}



\newSec{GPS}{3}

globale Ausrichtung




\newSec{Externe Sensoren und Mapping}{3}




\newSec{Dedektion der Umgebung}{2}

\missing[Auswertung von Objekten und Speicherung der zugehörigen Metadaten]



\newSec{Vorwärts-Berechnung}{3}







\newSec{Rückwärts-Berechnung bei Ringschluss}{3}

\missing[Position und Genauigkeit der vorherig ermittelten Objekte berechnen, wenn ein bekanntes Objekt erneut gesichtet wurde.]

Aufzeichnung des Weges erforderlich



\newSec{Sensoriken}{2}

\newSec{Abstandssensoren}{3}



\newSec{Punktuelle Abstandsmessung}{4}



\newSec{Linienförmige Abstandsmessung}{4}


\newSec{3D Abstandsmessung}{4}




























