\newSec[HWCOEX]{COEX Drohne}{2}

Bei dem für die \DHBW\ neu angeschafften \Quad\ handelt es sich um das Modell \Clover\ des Unternehmens \textit{Copter Express} (\COEX).

Das Modell \Clover\ wurde vom Hersteller zur Ausbildung und Forschung an \Quad[n] entwickelt. Das Modell besitzt einen Rahmen, welche die Rotoren bei Kollisionen schützten soll.




Interner Flight Controller, ROS Kommunikation via Pie 4.




Probleme bei Inbetriebnahme - Beispielprogramm des Herstellers bringt nicht das erwartete Ergebnis.





\newSec{Geometrie}{3}


\newSec[Geom-Rot]{Anordnung der Rotoren}{4}


X




\newSec[Control]{Control Stack}{3}

\missing[Bezeichnung auf deutsch?]

\newSec[PX]{Flight Controller}{4}
PX4 Racer


\newSec[Pie]{On Board Computer}{4}
\Pie\



\newSec{Funkfernsteuerung}{3}
s-Bus ??





\newSec{Sensorik}{3}



\newSec{Gyroskop}{4}



\newSec[Laser]{Laser-Abstandsmessung}{4}



\newSec{Bodenkamera}{4}


\newSec{GPS}{4}









\newSec[Build]{Aufbau des Bausatzes}{3}





\newSec{Aufbau}{4}

\missing[Bilder vom Aufbau]




\newSec[BuildTrouble]{Troubleshooting}{4}

\newSec{Platinenfehler}{5}




\newSec{Bus-System des RC Empfängers}{5}




\newSec[Build]{Inbetriebnahme}{3}


\newSec{Konfiguration des Flight Controllers}{4}




\newSec[TopicTrouble]{Troubleshooting}{4}


\missing[Topics anders benannt, als in Dokumentation (überall steht "/mavros" davor]

\missing[Achtung: der Regler VRA muss so eingestellt sein, dass "manual Flight" verfügbar ist. Andersfalls ist ein Start der Drohne nicht möglich.]


\newSec[TopicTroubleBlacklist]{Clover ist faul}{5}
Um Ressourcen zu sparen werden einige \Topic[s] nicht vom \Pie\ verarbeitet \bzw\ weitergegeben.

\texttt{Lösung}\\
Unterdrückte \Topic[s] müssen 
\missing[ungetested!]

https://clover.coex.tech/en/mavros.html

Freigegebene \Topic[s] in \texttt{mavros.launch}

\begin{lstlisting}[style=Style_XML, numbers=none, caption={[Freigegebene Topics] Freigegebene \Topic[s] in \texttt{mavros.launch}}]
<rosparam param="plugin_whitelist">
	- altitude
	- command
	- distance_sensor
	- ftp
	- global_position
	- imu
	- local_position
	- manual_control
	# - mocap_pose_estimate
	 - param
	 - px4flow
	 - rc_io
	 - setpoint_attitude
	- setpoint_position
	- setpoint_raw
	- setpoint_velocity
	- sys_status
	- sys_time
	- vision_pose_estimate
	# - vision_speed_estimate
	# - waypoint
</rosparam>
\end{lstlisting}


\newSec[TopicTroubleRCOverMD5]{md5-Sum des Topics OverrideRCIn}{5}
Nach erfolgreicher Kompilierung wird nachfolgender Laufzeitfehler ausgegeben, wenn sich ein \CodeClass{ros::Subscriber} oder ein \CodeClass{ros::Publisher} auf das \rTopic{/mavros/rc/override} anmeldet:\\
\textcolor{red}{[ERROR] [1643616625.226584828]: Client [/mavros] wants topic /mavros/rc/override \\to have datatype/md5sum [mavros\_msgs/OverrideRCIn/73b27a463a40a3eda1f9fbb1fc86d6f3], but our version has [mavros\_msgs/OverrideRCIn/fd1e1c08fa504ec32737c41f45223398]. Dropping connection.}

\texttt{Lösung}\\
Definition von \\\CodeMeth{struct MD5Sum<::mavros\_msgs::OverrideRCIn\_<ContainerAllocator>>} \\aus 

\begin{lstlisting}[style=Style_Bash, caption=Befehl zum Öffnen des \textit{OverrideRCIn}-Headers]
sudo nano /opt/ros/noetic/include/mavros_msgs/OverrideRCIn.h
\end{lstlisting}

\begin{lstlisting}[style=Style_CPP, numbers=none, caption=Definition des Struct \CodeStruct{MD5Sum} für das Template \textit{OverrideRCIn}]
template<class ContainerAllocator>
struct MD5Sum<::mavros_msgs::OverrideRCIn_<ContainerAllocator>>
{
	static const char* value()
	{
		return "73b27a463a40a3eda1f9fbb1fc86d6f3";
	}

	static const char* value(const ::mavros_msgs::OverrideRCIn_<ContainerAllocator>&)
	{
		return value();
	}
	
	static const uint64_t static_value1 = 0x73b27a463a40a3edULL;
	static const uint64_t static_value2 = 0xa1f9fbb1fc86d6f3ULL;
};
\end{lstlisting}

von dem \Pie\ kopieren und in der Definition des lokalen \ROS\ \textit{\mbox{OverrideRCIn}}-Headers ersetzen.



\newSec[TopicTroubleRCOver]{Topics OverrideRCIn läuft nur mit Topic RCIn}{5}





\texttt{Lösung}\\









\newSec[TopicTroubleRCOver]{Topics setpoint\_altitude/thrust läuft nicht}{3}





\newSec{Lösung}{4}






\newSec[Build]{Mögliche Lösung der Aufgabenstellung}{3}



\newSec[COEXPkg]{COEX-Package}{4}
Für die Interaktion mit der COEX-Drohne wurden diverse Klassen erstellt, um einzelne Aspekte der Interaktion mit der Drohne umsetzen zu können.





\newSec[COEXPkg]{geeignete Topics}{4}

Mit dem \rTopic{blablub} \missing\ kann eine Regelung in der XY-Ebene umgesetzt werden. Die Höhenregelung kann mit set\_attitude/thrust eingeführt werden.
Somit wäre der Versuch für nachfolgende Studierende sehr viel sicherer und so.


