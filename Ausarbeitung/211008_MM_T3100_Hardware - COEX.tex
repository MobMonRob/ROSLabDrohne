\newSec[HWCOEX]{COEX Drohne}{2}

Bei dem für die \DHBW\ neu angeschafften \Quad\ handelt es sich um das Modell \Clover\ des Unternehmens \textit{Copter Express} (\COEX).

Das Modell \Clover\ wurde vom Hersteller zur Ausbildung und Forschung an \Quad[n] entwickelt. Das Modell besitzt einen Rahmen, welche die Rotoren bei Kollisionen schützten soll.




Interner Flight Controller, ROS Kommunikation via Pie 4.




Probleme bei Inbetriebnahme - Beispielprogramm des Herstellers bringt nicht das erwartete Ergebnis.




\newSec[Control]{Control Stack}{3}
Als \textit{Flight Controller} wird das Modell \textit{PX4 Racer} genutzt. Die Firmware, sowie weitere Software zur Interaktion mit der Drohne \Clover\ werden von \textit{Dronecode Foundation} bereitgestellt.

Die Anbindung an \ROS wird durch einen \Pie\ realisiert. Hierbei wird der On-Board Computer als \textit{roscore} genutzt.


\newSec{Funkfernsteuerung}{3}
s-Bus ??





\newSec{Sensorik}{3}

\missing[EIn bisschen Einleitung.]


\begin{itemize}
\item Gyroskop
\item Laser-Abstandsmessung zum Boden
\item GPS
\item Bodenkamera
\end{itemize}










\newSec[Build]{Aufbau des Bausatzes}{3}





\newSec{Aufbau}{4}

\missing[Bilder vom Aufbau]








\newSec[Build]{Inbetriebnahme}{3}


\newSec{Konfiguration des Flight Controllers}{4}



\newSec{Testflug}{4}



\comp{coexExample}
\missing[Verweis auf Example Code]













\newSec[Build]{Mögliche Lösung der Aufgabenstellung}{3}



\newSec[COEXPkg]{COEX-Package}{4}
Für die Interaktion mit der \COEX-Drohne wurden diverse Klassen erstellt, um einzelne Aspekte der Interaktion mit der Drohne umsetzen zu können.

Nach dem Wechsel auf die andere Drohne wurde die Aktualisierung dieses Package nicht weiter verfolgt. Sofern eine Einbindung der \COEX-Drohne in die Ergebnisse dieser \Arbeit\ durchgeführt werden soll, müss dieses Package entsprechend angepasst werden.



\newSec[COEXPkg]{geeignete Topics}{4}

Mit dem \rTopic{blablub} \missing\ kann eine Regelung in der XY-Ebene umgesetzt werden. Die Höhenregelung kann mit set\_attitude/thrust eingeführt werden.
Somit wäre der Versuch für nachfolgende Studierende sehr viel sicherer und so.


\missing\



\newSec[CoexLiteratur]{hilfreiche Literatur}{4}
Nachfolgend sollen Internetseiten genannt werden, welche die Einarbeitung in den Umgang mit der \COEX-Drohne vereinfachen können.

\begin{itemize}
\item https://clover.coex.tech/en/wifi.html

\item https://clover.coex.tech/en/simple\_offboard.html
\item https://docs.px4.io/master/en/ros/mavros\_offboard.html
\item https://docs.px4.io/master/en/flight\_modes/offboard.html

\item https://mavlink.io/en/services/manual\_control.html
\item http://wiki.ros.org/mavros\#mavros.2FPlugins.manual\_control

\item https://mavlink.io/en/messages/common.html\#SET\_POSITION\_TARGET\_LOCAL\_NED
\end{itemize}

Spezifische Verweise sind im Quellcode des \COEX-Package hinterlegt.



\newSec[COEXTrouble]{Troubleshooting}{4}



\missing[Achtung: der Regler VRA muss so eingestellt sein, dass "manual Flight" verfügbar ist. Andersfalls ist ein Start der Drohne nicht möglich.]





\newSec{Platinenfehler}{5}

\missing[Hier anmerken, dass das Problem bisher nicht behoben wurde => Wirkt sich nur auf LED-Streifen aus.]





\newSec{Bus-System des RC Empfängers}{5}
\missing[RC-Empfänger gibt per default i-Bus aus, PX4 erwartet s-Bus.]

\missing[Bild von Oszilloskop]


\texttt{Lösung}\\
\missing[i-Bus und s-Bus werden durch Halten des Knopfens getauscht. Verweis auf Homepage angeben?]





\newSec[TopicTroubleRCOverMD5]{md5-Sum des Topics OverrideRCIn}{5}
Nach erfolgreicher Kompilierung wird nachfolgender Laufzeitfehler ausgegeben, wenn sich ein \CodeClass{ros::Subscriber} oder ein \CodeClass{ros::Publisher} auf das \rTopic{/mavros/rc/override} anmeldet:\\
\textcolor{red}{[ERROR] [1643616625.226584828]: Client [/mavros] wants topic /mavros/rc/override \\to have datatype/md5sum [mavros\_msgs/OverrideRCIn/73b27a463a40a3eda1f9fbb1fc86d6f3], but our version has [mavros\_msgs/OverrideRCIn/fd1e1c08fa504ec32737c41f45223398]. Dropping connection.}

\texttt{Lösung}\\
Definition von \\\CodeMeth{struct MD5Sum<::mavros\_msgs::OverrideRCIn\_<ContainerAllocator>>} \\aus 

\begin{lstlisting}[style=Style_Bash, caption=Befehl zum Öffnen des \textit{OverrideRCIn}-Headers]
sudo nano /opt/ros/noetic/include/mavros_msgs/OverrideRCIn.h
\end{lstlisting}

\begin{lstlisting}[style=Style_CPP, numbers=none, caption=Definition des Struct \CodeStruct{MD5Sum} für das Template \textit{OverrideRCIn}]
template<class ContainerAllocator>
struct MD5Sum<::mavros_msgs::OverrideRCIn_<ContainerAllocator>>
{
	static const char* value()
	{
		return "73b27a463a40a3eda1f9fbb1fc86d6f3";
	}

	static const char* value(const ::mavros_msgs::OverrideRCIn_<ContainerAllocator>&)
	{
		return value();
	}
	
	static const uint64_t static_value1 = 0x73b27a463a40a3edULL;
	static const uint64_t static_value2 = 0xa1f9fbb1fc86d6f3ULL;
};
\end{lstlisting}

von dem \Pie\ kopieren und in der Definition des lokalen \ROS\ \textit{\mbox{OverrideRCIn}}-Headers ersetzen.
Ein Versuch, den \Pie\ einem Update zu unterziehen, ist fehlgeschlagen. Somit ist eine unmittelbare Synchronisation der Nachrichten-Typen aufwändig.


