\newSec[ImplApp]{Application Layer - \textit{DroneController} Package}{2}

\newSec{Batteryable}{4}
Die \CodeClass{Batteryable} hat die Aufgabe, den Akku-Stand des \Quad\ zu überwachen. Die Klasse bietet die virtuellen Methoden \CodeMeth{getVoltage()} und \CodeMeth{getPercentage()} an, die ableitende Klasse kann die Methoden-Definition in abhängigkeit der von der Hardware zur Verfügung gestellten Daten überschreiben.


\newSec{DroneControlable}{4}
Um die Interaktion mit der Hardware gekapselt zu gestalten und zudem die korrekte Initialisierung der beteiligten Klassen zu gewährleisten, wird die \CodeClass{DroneControlable} als Basis für eine Fassade eingesetzt. Die Klasse selbst erhält Pointer auf die für die Funktionalität des \Quad[s] benötigten Klassen (\CodeClass{Batteryable}, \CodeClass{PoseControlable}, \CodeClass{IMUState}, \CodeClass{Statusable} und \CodeClass{Transmitable}).\\
Im Konstruktor der \CodeClass{DroneControlable} werden die Sicherheitsfunktionalitäten der verwalteten Instanzen miteinander verknüpft.


\newSec{IMUable}{4}
Die \CodeClass{IMUable} wurde als Basis-Klasse für die Aufnahme der \textit{Inertial Measurement Unit} (\textit{IMU}) Daten und die Weiterleitung zur Berechnung und Regelung der Pose implementiert.\\
Neben der Sicherheitsfunktionalität (Ermittlung von Zusammenstößen anhand eines Beschleunigungsschwellwerts) werden als Attribute Pointer auf Instanzen der  \CodeClass{PoseBuildable} und \CodeClass{PoseControlable} eingesetzt. Hierbei handelt es sich im \textit{Dependency Injections}.


\newSec{IMUState}{4}
Die Klasse \CodeClass{IMUState} beinhaltet Attribute, welche die Daten der \textit{Inertial Measurement Unit} (\textit{IMU}) abbildet.\\
\note{Mit der Änderung der Hardware ergab sich, dass die \CodeClass{IMUState} Daten der lokalen Orientierung enthält. Der \Quad\ \Clover\ sendet die Rate der Winkeländerung.}\\
Im Sinne der Auswertung und Parameter-Optimierung wurde hier zusätzlich der gemessene Abstand zum Untergrund als Attribut aufgenommen. Dieser dient nur zu Referenzzwecken und wird, entsprechend der Aufgabenstellung (siehe \refCap{Problemstellung}), nicht zur Berechnung der Pose eingesetzt.


\newSec{Pose}{4}
Die \CodeClass{Pose} bildet die in \refCap{Pose} definierten Eingeschaften ab. Hierzu weden Instanzen der \CodeClass{Vector3D} eingesetzt. Zudem erhält jede berechente Pose einen Zeitstempel. Hierüber kann in einer Weiterführung des Projekts ein Flugweg zeitlich korrekt wiedergegeben werden.\\
Für eine zukünftige Erweiterung sind bereits Attribute zur Erfassung der Unsicherheit der berechneten Pose vorhanden. Die Unsicherheiten können für Mapping-Algorithmen genutzt werden.


\newSec{PoseBuildable}{4}
Die \CodeClass{PoseBuildable} bietet eine Basis für \textit{Dependency Inversion} einer Factory für die \CodeClass{Pose}. Sie ist in diesen Paket angesiedelt, um der \CodeClass{IMUable} Zugriff zu ermöglichen. Eine Implementierung der Funktionalität wird in der PlugIn-Schicht übernommen.


\newSec{PoseControlable}{4}
Ebenfalls als Basis für eine \textit{Dependency Inversion} wird die \CodeClass{PoseControlable} genutzt. Nach der Berechnung der aktuellen Pose aus den aufgearbeiteten Beschleunigungsdaten wird die Pose an die Implementierung der \CodeClass{PoseControlable} übertragen. Hier werden die Stellgrößen für die Regelung des \Quad[s] ermittelt und anschließend die \CodeMeth{transmitAction} der \CodeClass{Transmitable} aufgerufen.


\newSec{Statusable}{4}
Die Aufgabe der \CodeClass{Statusable} ist die Interaktion mit den Status des \Quad\ und das senden von Befehlen, welche den Zustandsautomaten des \Quad[s] beeinflussen. Als Maßgebliche Befehle werde Start- und Lande-Befehle übertragen.


\newSec{Timable}{4}
Um eine allgemeine Zeit innerhalb des Systeme definieren zu können, wurde die \CodeClass{Timeable} implementiert. Die Methoden erlauben eine Ermittlung der vergangenen Zeit ab Programmstart \bzw\ erster eingetragener Nachricht.\\
\note{Die Zeitstempel der mittels \ROS\ übertragenen Nachrichten enthalten die Zeit entsprechend \href{https://unixtime.org/}{Unix Epoch Time}}


\newSec{Transmitable}{4}
Die \CodeClass{Transmitable} bietet als \textit{full virutal} Klasse die \CodeMeth{bool transmitAction(double roll, double pitch, double yarn, double thrust)} zur Implementierung an. Von dieser Klasse ableitende Klassen übernehmen die Aufgabe, Steuerungsbefehle an die Hardware zu übersenden.


\newSec{ImpactOK und ImpactWatcher}{4}
Als Sicherheitsfunktionalität wird der \CodeClass{IMUable} eine Instanz der \CodeClass{ImpactWatcher} übergeben. Diese erhält zur Prüfung der Sicherheit Beschleunigungsdaten, welche durch die \CodeClass{StateBuilder} aufgearbeitet wurden. Überschreitet der Betrag der Beschleunigungswerte einen Schwellwert, wird die Sicherheitsfunktionalität ausgelöst.\\
\note{Die vorgeschlagene Vorgehensweise kann nur angewandt werden, wenn der \ROS-\textit{core} auf dem Rechner (nicht auf dem \Quad) ausgeführt wird.}


\newSec{PercentageOK und VoltageOK}{4}
Die Ableitungen der \CodeClass{SafetRequirement} überprüfen, ob die als Pointer übergebene Instanz der \CodeClass{Batteryable} die geprüfte Eigenschaft (Spannung oder Akkustand) einhält.


\newSec{TimeoutOK und WatchDog}{4}
Wie aus der Namensgebung ableitbar sind diese Klassen dafür gedacht, das System über Ausfälle der Datenübertragung zu informieren.
Der derzeitige Stand prüft den zeitlichen Unerscheid zwischen eingegangenen Nachrichten, angewandt in der Instanz der \CodeClass{Statusable}.
Als Verbesserung für diese Funktionalität kann ein nebenläufiger Timer implementiert werden, welcher bei Ablauf die Sicherheitsfunktionalität auslöst. Eine eingehende Nachricht würde den Timer zurücksetzen.


\newSec{Rotor und Thrustable}{4}
Die \CodeClass{Rotor} wurde versuchsweise hinzugefügt. Unter Umständen kann hier die auf den \Quad\ wirkenden Kräfte ermittelt werden und somit die Regelung in einer Umgebung ohne äußere Einflüsse verbessert werden.\\
Als Interkation mit einem \Quad\ wird die virtuelle \CodeClass{Thrustable} eingeführt. Hier werden die vier Rotoren (Instanzen der \CodeClass{Rotor}) und die zugehörige Geometrie des \Quad[s] hinterlegt. Aus den Drehzahlen der Rotoren lässt sich die wirkende Gesamtkraft ermitteln.\\
\note{Diese Klassen finden im aktuellen Programm keine Verwendung.}


