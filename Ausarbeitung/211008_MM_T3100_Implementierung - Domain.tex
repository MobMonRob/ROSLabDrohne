\newSec[ImplDomain]{Domain Layer}{2}

Der \textit{Domain Layer} stellt als Teil der \clean\ die Ebene dar, in der allgemein gültige Typen oder Definitionen abgelegt werden können. Nachfolgend werden die Klassen beschrieben, die in diesem Projekt zu dem \textit{Domain Layer} gehören. Sofern nicht anders betitelt, handelt es sich bei den Klassen um Klassen vom Typ \VO.


\newSec{Optional}{4}
Die \CodeClass{Optional} soll dem Ringbuffer ermöglichen, das Nichtvorhandensein von Einträgen darstellen zu können. Diese Klasse ist als \textit{template} implementiert und enthält neben dem Speicherplatz für die generische Instanz einen bool'schen Wert als Validierung.
Hiermit wird das Arbeit mit \textit{Null-pointern} im Kontext desr \CodeClass{Ringbuffer} vermieden.
\note{Mit der Verwendung von c++17 ist eine vergleichbare Klasse Teil der Standard-Bibliothek. Die Sprachversion des Projekts ist c++11.}


\newSec{Ringbuffer}{4}
Die \CodeClass{Ringbuffer} soll einen Ringspeicher abbilden. Hier wird nicht -wie ner Name vermuten lässt- ein Ringspeicher im Sinne einer ringförmig verketteten Liste implementiert. Diese Klasse Kapselt eine \textit{Standard Template Library} (\textit{STL}) vom Typ \CodeClass{std::vector<T>}, wobei bei Überschreiten der maximalen Anzahl an Elementen das vordere Elemente entfernt wird.\\
\note{Die Implementierung der \CodeClass{std::vector<T>} sieht vor, neue Elemente an das Ende anzuhängen.}


\newSec{TimedValue}{4}
Die \CodeClass{TimedValue} soll einen mit einem Zeitstemlep versehenen Wert abbilden. Dies wird durch das Erben von den Klassen \CodeClass{Timestamp} und \CodeClass{Value} umgesetzt.


\newSec{Timestamp}{4}
Mit der \CodeClass{Timestamp} wird ein Zeitstempel eingeführt. Alle \ROS-Nachrichten beinhalten durch die Kapselung der \CodeClass{std\_msgs::Header}-Klasse einen Zeitstempel.


\newSec{Unit}{4}
Mit \CodeClass{Unit} werden Einheiten umgesetzt, um eine korrekte Übergabe von \CodeClass{Value}-Instanzen zur Laufzeit zu gewährleisten.


\newSec{Value}{4}
Die \CodeClass{Value} bildet einen Wert ab. Sie besteht aus einer \CodeClass{Unit} und einem dazugehöigen Zahlenwert.


\newSec{Vector3D}{4}
Bei der \CodeClass{Vector3D} handelt es sich um die Abbildung eines Vektors im dreidimensionalen Raum. Zusätzlich wird dem Vektor eine Einheit zugewiesen.
Zudem werden in dieser Klasse grundlegende mathematische Operationen für Vektoren implementiert.


\newSec{SafetyProvider}{4}
Die \CodeClass{SafetyProvider} ist in ihrer Funktionalität abgeschlossen.
Instanzen der \CodeClass{SafetRequirement} werden an einen \textit{SafetyProvider} übergeben. auf Anfrage wird geprüft, ob die übergebenen Instanzen der \CodeClass{SafetRequirement} die erwartungen erfüllen. Ist dies nicht der Fall, wird die \CodeMeth{safetyTriggered} der hinterlegten Instanzen der \CodeClass{SafetReceiver} aufgerufen.


\newSec{SafetReceiver}{4}
Die \textit{full virtual} \CodeClass{SafetReceiver} fordert die Implementierung der \CodeMeth{safetyTriggered}. Hierin definiert die erbende Klasse, wie in Folge einer Verletzung der Sicherheitsanforderungen reagiert werden soll. Allgemein ist hierfür eine Landung des \Quad[s] vorgesehen.


\newSec{SafetRequirement}{4}
Im Sinne des \textit{Open/Close-Principle} wird die \CodeClass{SafetRequirementy} als virtuelle Klasse eingeführt. Hierdurch werden Eigenschaften und Status \bzw\ Flags des \Quad[s] für Sicherheitsfunktionalitäten zugänglich gemacht.


\newSec{SafetyTranslative}{4}
Eine Instanz der \CodeClass{SafetyTranslative} erhält eine Referenz auf eine Instanz der \CodeClass{SafetyProvider}. Ziel dieser Klasse ist es, die Sicherheitsfunktionalität eines anderen \textit{SafetyProvider} als \textit{SafetRequirementy} einbinden zu können.


