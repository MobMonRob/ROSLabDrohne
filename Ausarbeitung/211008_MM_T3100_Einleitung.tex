\newSec{Einleitung}{1}


\vspace{1cm}
\begin{center}
\textbf{\grqq Über den Wolken muss die Freiheit wohl grenzenlos sein.\grqq}\\
\vspace{0.5cm}
aus dem Lied \textit{Über den Wolken} von Reinhard Mey, 1973 \cite{Einl1}
\end{center}
\vspace{1cm}


Schon lange sehnten sich Menschen danach, fliegen zu können. Eine der bekanntesten Geschichten ist die Mythologie über Ikarus, der auf der Flucht aus dem Labyrinth auf der Insel Kreta mit selbstgebauten Flügeln der Sonne zu nahe kam und daraufhin ins Meer stürzte. \comp{Einl2}\\
Leonardo da Vinci deutete Ende des 15. Jahrhunderts in den \textit{Pariser Manuskripten} die Funktionsweise eines Helikopters \bzw\ einer Luftschraube an. \comp{Einl3}

Diverse Paket-Lieferdienste planen, \Quad\ in Ballungsgebieten zum Versand von Paketen einzusetzen. \comp{Einl4}

Zur Wegplanung ist eine korrekt berechnete Pose (Position und Orientierung) des \Quad[s] notwendig. Hierzu können diverse Sensoren eingesetzt werden, welche an den Fuggeräten eingesetzt werden.\\
In dieser \Arbeit\ wurden Initialsensoren als mögliche Herangehensweise ausgewählt. Als Grundlage für die Auswahl kann die einfache Aufarbeitung der Daten genannt werden. Zudem wurden bereits erfolgreich Projekte mit diesem Konzept umgesetzt. \compB{Paper3}{Paper4}






\vspace{3.5cm}
\note{Für diese Ausarbeitung werden fachliche Begrifflichkeiten vorausgesetzt, sofern diese nicht innerhalb der Ausarbeitung erklärt werden. Sind Begriffe für Lesende unklar, sind diese an geeigneter Stelle nachzuschlagen. Auf eine voranstehende Erklärung aller genutzten und nicht näher erklärten Begriffe wird in dieser Ausarbeitung verzichtet, um den Rahmen dieser Arbeit einhalten zu können.}







