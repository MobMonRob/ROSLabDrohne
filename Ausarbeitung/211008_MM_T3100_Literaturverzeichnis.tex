\clearpage
\addcontentsline{toc}{section}{Literaturverzeichnis}
\newcounter{IndexLiteratur}
\newcommand{\NewBibItem}[2]{\stepcounter{IndexLiteratur} \bibitem[\theIndexLiteratur]{#1} #2,\newline}

% Layout Commands
\newcommand{\Published}[1]{veröffentlicht #1}
\newcommand{\Modified}[1]{verändert #1}
\newcommand{\Requested}[1]{abgefragt #1}
\newcommand{\Online}[1]{online,  #1\newline}
\newcommand{\Version}[1]{#1. Auflage}
\newcommand{\ISBN}[1]{ISBN #1}
\newcommand{\Article}[1]{Artikel \textit{#1}}
\newcommand{\Chapter}[1]{Kapitel \textit{#1}}
\newcommand{\Pages}[1]{Seite #1}
\newcommand{\NA}{-unbekannt-}



\begin{thebibliography}{\theIndexLiteratur}

% Paper
\NewBibItem{Paper1}{Pozo D., Romero L., Rosales J., Quadcopter stabilization by using PID controllers}
\Published{2014}

\NewBibItem{Paper2}{Fresk E., Nikolakopoulos G., Full Quaternion Based Attitude Control for a Quadrotor}
\Published{19.07.2013}



% Bücher
\NewBibItem{RT1}{Heinrich B. (Hrsg.), et al., Kaspers/KOfner Messen - Steuern - Regeln}
\Version{8., überarbeitete und ergänzte Auflage}
\Published{2009}
\ISBN{978-3-8348-0006-0}

\NewBibItem{SV1}{Wendemuth A., Grundlagen der digitalen Signalverarbeitung}
\Published{2005}
\ISBN{3-540-21885-8}





% Normen
\NewBibItem{IEC60050-351}{Internationales Elektrotechnisches Wörterbuch – Teil 351: Leittechnik (IEC 60050-351:2006)}
\Published{06/2009}


% Einleitung
\NewBibItem{Einl1}{Mey R., Reinhard Mey Textsammlung 14.Auflage}
\Online{https://www.reinhard-mey.de/texte-fuer-alle/}
\Published{13.12.2017}, \Requested{08.02.2022}







% Problemstellung
\NewBibItem{Ballon}{ellwangen2010, Ballon steuern?}
\Online{https://www.ballonfahrten.com/ballon-steuern/}
\Published{20.02.2010}, \Requested{07.11.2021}



% Regelungstechnik
\NewBibItem{ContrD}{D-Glied}
\Online{http://testcon.info/FB\_DE\_D-Glied.html}
\Published{\NA}, \Requested{13.04.2022}











% ROS
\NewBibItem{ros1}{ROS - Robot Operating System}
\Online{https://www.ros.org}
\Published{\NA}, \Requested{28.11.2021}

\NewBibItem{rosVersion}{ROS Indigo Igloo}
\Online{http://wiki.ros.org/indigo}
\Published{\NA}, \Modified{08.01.2018}, \Requested{16.03.2022}







%Hardware
\NewBibItem{Quad1}{How quadcopters work \& fly: An intro to multirotors}
\Online{https://www.droneybee.com/how-quadcopters-work/}
\Published{\NA}, \Modified{20.11.2017}, \Requested{01.04.2022}




%Software





%COEX
\NewBibItem{coexExample}{MAVROS Offboard control example}
\Online{https://docs.px4.io/master/en/ros/mavros\_offboard.html}
\Published{\NA}, \Modified{02.02.2021}, \Requested{16.03.2022}






%Parrot
\NewBibItem{parotSDK}{AR.Drone Developer Guide}
\Chapter{AR.Drone 2.0 Overview}, \Pages{5 ff.}
\Online{https://jpchanson.github.io/ARdrone/ParrotDevGuide.pdf}
\Published{21.05.2012}, \Requested{17.03.2022}










%Language
\NewBibItem{cpp1}{Saks D., Better even at the lowest levels}
\Online{https://www.embedded.com/better-even-at-the-lowest-levels/}
\Published{01.11.2008}, \Modified{05.12.2020}, \Requested{28.07.2021}

\NewBibItem{c1}{Application Note Object-Oriented Programming in C}
\Online{https://www.state-machine.com/doc/AN\_OOP\_in\_C.pdf}
\Published{06.11.2020}, \Requested{28.07.2021}

\NewBibItem{DyString}{Kirk N., How do strings allocate memory in c++?}
\Online{https://stackoverflow.com/questions/18312658/how-do-strings-allocate-memory-in-c}
\Published{19.08.2013}, \Requested{17.08.2021}

\NewBibItem{Dy1}{Bansal A., Containers in C++ STL (Standard Template Library)}
\Online{https://www.geeksforgeeks.org/containers-cpp-stl/}
\Published{05.03.2018}, \Modified{12.07.2020}, \Requested{17.08.2021}

\NewBibItem{DefASA}{Automatic Storage Duration}
\Online{https://www.oreilly.com/library/view/c-primer-plus/9780132781145/ch09lev2sec2.html}
\Published{\NA}, \Requested{17.08.2021}

\NewBibItem{DefDSA}{Noar J., Orda A., Petruschka Y., Dynamic storage allocation with known durations}
\Online{https://www.sciencedirect.com/science/article/pii/S0166218X99001754}
\Published{30.03.2000}, \Requested{17.08.2021}




%Varianten




\end{thebibliography}

\Anmerkung{Wird hier ein Veröffentlichungsdatum als \grqq -unbekannt-\grqq\ markiert, so konnte diese Angabe weder auf der entsprechenden Webseite, noch in deren Quelltext ausfindig gemacht werden.}

\clearpage